% Options for packages loaded elsewhere
\PassOptionsToPackage{unicode}{hyperref}
\PassOptionsToPackage{hyphens}{url}
%
\documentclass[
]{article}
\usepackage{amsmath,amssymb}
\usepackage{iftex}
\ifPDFTeX
  \usepackage[T1]{fontenc}
  \usepackage[utf8]{inputenc}
  \usepackage{textcomp} % provide euro and other symbols
\else % if luatex or xetex
  \usepackage{unicode-math} % this also loads fontspec
  \defaultfontfeatures{Scale=MatchLowercase}
  \defaultfontfeatures[\rmfamily]{Ligatures=TeX,Scale=1}
\fi
\usepackage{lmodern}
\ifPDFTeX\else
  % xetex/luatex font selection
\fi
% Use upquote if available, for straight quotes in verbatim environments
\IfFileExists{upquote.sty}{\usepackage{upquote}}{}
\IfFileExists{microtype.sty}{% use microtype if available
  \usepackage[]{microtype}
  \UseMicrotypeSet[protrusion]{basicmath} % disable protrusion for tt fonts
}{}
\makeatletter
\@ifundefined{KOMAClassName}{% if non-KOMA class
  \IfFileExists{parskip.sty}{%
    \usepackage{parskip}
  }{% else
    \setlength{\parindent}{0pt}
    \setlength{\parskip}{6pt plus 2pt minus 1pt}}
}{% if KOMA class
  \KOMAoptions{parskip=half}}
\makeatother
\usepackage{xcolor}
\usepackage[margin=1in]{geometry}
\usepackage{color}
\usepackage{fancyvrb}
\newcommand{\VerbBar}{|}
\newcommand{\VERB}{\Verb[commandchars=\\\{\}]}
\DefineVerbatimEnvironment{Highlighting}{Verbatim}{commandchars=\\\{\}}
% Add ',fontsize=\small' for more characters per line
\usepackage{framed}
\definecolor{shadecolor}{RGB}{248,248,248}
\newenvironment{Shaded}{\begin{snugshade}}{\end{snugshade}}
\newcommand{\AlertTok}[1]{\textcolor[rgb]{0.94,0.16,0.16}{#1}}
\newcommand{\AnnotationTok}[1]{\textcolor[rgb]{0.56,0.35,0.01}{\textbf{\textit{#1}}}}
\newcommand{\AttributeTok}[1]{\textcolor[rgb]{0.13,0.29,0.53}{#1}}
\newcommand{\BaseNTok}[1]{\textcolor[rgb]{0.00,0.00,0.81}{#1}}
\newcommand{\BuiltInTok}[1]{#1}
\newcommand{\CharTok}[1]{\textcolor[rgb]{0.31,0.60,0.02}{#1}}
\newcommand{\CommentTok}[1]{\textcolor[rgb]{0.56,0.35,0.01}{\textit{#1}}}
\newcommand{\CommentVarTok}[1]{\textcolor[rgb]{0.56,0.35,0.01}{\textbf{\textit{#1}}}}
\newcommand{\ConstantTok}[1]{\textcolor[rgb]{0.56,0.35,0.01}{#1}}
\newcommand{\ControlFlowTok}[1]{\textcolor[rgb]{0.13,0.29,0.53}{\textbf{#1}}}
\newcommand{\DataTypeTok}[1]{\textcolor[rgb]{0.13,0.29,0.53}{#1}}
\newcommand{\DecValTok}[1]{\textcolor[rgb]{0.00,0.00,0.81}{#1}}
\newcommand{\DocumentationTok}[1]{\textcolor[rgb]{0.56,0.35,0.01}{\textbf{\textit{#1}}}}
\newcommand{\ErrorTok}[1]{\textcolor[rgb]{0.64,0.00,0.00}{\textbf{#1}}}
\newcommand{\ExtensionTok}[1]{#1}
\newcommand{\FloatTok}[1]{\textcolor[rgb]{0.00,0.00,0.81}{#1}}
\newcommand{\FunctionTok}[1]{\textcolor[rgb]{0.13,0.29,0.53}{\textbf{#1}}}
\newcommand{\ImportTok}[1]{#1}
\newcommand{\InformationTok}[1]{\textcolor[rgb]{0.56,0.35,0.01}{\textbf{\textit{#1}}}}
\newcommand{\KeywordTok}[1]{\textcolor[rgb]{0.13,0.29,0.53}{\textbf{#1}}}
\newcommand{\NormalTok}[1]{#1}
\newcommand{\OperatorTok}[1]{\textcolor[rgb]{0.81,0.36,0.00}{\textbf{#1}}}
\newcommand{\OtherTok}[1]{\textcolor[rgb]{0.56,0.35,0.01}{#1}}
\newcommand{\PreprocessorTok}[1]{\textcolor[rgb]{0.56,0.35,0.01}{\textit{#1}}}
\newcommand{\RegionMarkerTok}[1]{#1}
\newcommand{\SpecialCharTok}[1]{\textcolor[rgb]{0.81,0.36,0.00}{\textbf{#1}}}
\newcommand{\SpecialStringTok}[1]{\textcolor[rgb]{0.31,0.60,0.02}{#1}}
\newcommand{\StringTok}[1]{\textcolor[rgb]{0.31,0.60,0.02}{#1}}
\newcommand{\VariableTok}[1]{\textcolor[rgb]{0.00,0.00,0.00}{#1}}
\newcommand{\VerbatimStringTok}[1]{\textcolor[rgb]{0.31,0.60,0.02}{#1}}
\newcommand{\WarningTok}[1]{\textcolor[rgb]{0.56,0.35,0.01}{\textbf{\textit{#1}}}}
\usepackage{longtable,booktabs,array}
\usepackage{calc} % for calculating minipage widths
% Correct order of tables after \paragraph or \subparagraph
\usepackage{etoolbox}
\makeatletter
\patchcmd\longtable{\par}{\if@noskipsec\mbox{}\fi\par}{}{}
\makeatother
% Allow footnotes in longtable head/foot
\IfFileExists{footnotehyper.sty}{\usepackage{footnotehyper}}{\usepackage{footnote}}
\makesavenoteenv{longtable}
\usepackage{graphicx}
\makeatletter
\def\maxwidth{\ifdim\Gin@nat@width>\linewidth\linewidth\else\Gin@nat@width\fi}
\def\maxheight{\ifdim\Gin@nat@height>\textheight\textheight\else\Gin@nat@height\fi}
\makeatother
% Scale images if necessary, so that they will not overflow the page
% margins by default, and it is still possible to overwrite the defaults
% using explicit options in \includegraphics[width, height, ...]{}
\setkeys{Gin}{width=\maxwidth,height=\maxheight,keepaspectratio}
% Set default figure placement to htbp
\makeatletter
\def\fps@figure{htbp}
\makeatother
\setlength{\emergencystretch}{3em} % prevent overfull lines
\providecommand{\tightlist}{%
  \setlength{\itemsep}{0pt}\setlength{\parskip}{0pt}}
\setcounter{secnumdepth}{-\maxdimen} % remove section numbering
\ifLuaTeX
  \usepackage{selnolig}  % disable illegal ligatures
\fi
\IfFileExists{bookmark.sty}{\usepackage{bookmark}}{\usepackage{hyperref}}
\IfFileExists{xurl.sty}{\usepackage{xurl}}{} % add URL line breaks if available
\urlstyle{same}
\hypersetup{
  hidelinks,
  pdfcreator={LaTeX via pandoc}}

\author{}
\date{\vspace{-2.5em}}

\begin{document}

\hypertarget{introduction}{%
\subsection{Introduction}\label{introduction}}

L'équipe de projet
\texttt{The\ World\ (Taylor\textquotesingle{}s\ Version)} est composée
de quatres personnes : Marielle CHARON, Kevin HERNANDEZ, Anouchka NEVEU
et Amadou ISSAKA AMADOU. Le but de ce projet est de répondre à plusieurs
questions sur le thème de la musique et plus précisement sur les
platformes Spotify et Youtube grace à la visualisation de données. Le
projet sera développé sous R.

\hypertarget{donnuxe9es}{%
\subsection{Données 💡}\label{donnuxe9es}}

Le dataset utilisé (\texttt{Datasetfinal.csv}) est un merge de deux
datasets :
\href{https://www.kaggle.com/datasets/salvatorerastelli/spotify-and-youtube}{Spotify
and Youtube} et
\href{https://www.kaggle.com/datasets/sujaykapadnis/spotify-songs}{Spotify
songs}. Le merge final pèse 6.55 MB et est sous format CSV. Il contient
3668 éléments. La clé qui a servi au merge est l'ID de la musique sur
Spotify.

Les deux jeux de données contiennent des données datant de \textbf{2023}
collectées depuis les API officielles de YouTube et Spotify. On a,
notamment, les 10 musiques les plus populaires d'une variété d'artistes.

\begin{quote}
Il possède à l'heure actuelle 50 features, avant cleaning de celles
pertinentes. On regroupe les features sous ces catégories :
\end{quote}

\begin{longtable}[]{@{}
  >{\raggedright\arraybackslash}p{(\columnwidth - 2\tabcolsep) * \real{0.5000}}
  >{\raggedright\arraybackslash}p{(\columnwidth - 2\tabcolsep) * \real{0.5000}}@{}}
\toprule\noalign{}
\begin{minipage}[b]{\linewidth}\raggedright
Catégorie
\end{minipage} & \begin{minipage}[b]{\linewidth}\raggedright
Features associées
\end{minipage} \\
\midrule\noalign{}
\endhead
\bottomrule\noalign{}
\endlastfoot
Informations générales & artist, track\_name, album, album\_type,
duration\_ms, channel, description, track\_album\_release\_date,
\ldots{} \\
Scores musicaux & danceability, energy, key, loudness, acousticness,
instrumentalness, liveness, valence, tempo \\
Popularité & views, likes, comments, stream, track\_popularity \\
Variables supplémentaires & uri (Spotify ID), url\_spotify,
url\_youtube, licensed, \ldots{} \\
\end{longtable}

\begin{quote}
Les variables supplémentaires n'expliquent pas la donnée pour l'analyse,
mais sont utilisées pour structurer le dataset. Les features se
répartissent selon plusieurs types : nominales (nom de
l'artiste/musique/album, \ldots), discrètes (tonalité) ou continues
(scores musicaux).
\end{quote}

\begin{center}\rule{0.5\linewidth}{0.5pt}\end{center}

\hypertarget{ruxe9ponses-aux-questions}{%
\subsection{Réponses aux questions 🤔}\label{ruxe9ponses-aux-questions}}

\begin{quote}
Quelles sont les chansons les plus populaires sur YouTube et Spotify ?
\end{quote}

\begin{quote}
Y a-t'il une corrélation entre les genres de musique et leur clé
musicale ?
\end{quote}

Dans le graphique ci-dessous, on voit les 15 chansons les plus écoutées
(streams cumulés sur YouTube et Spotify). On voit rapidement que
\emph{Ed Sheeran} domine le classement puisqu'il cumule 3 des 15
chansons les plus écoutées. ⚠️ \textbf{CELA NE REPRESENTE PAS LES
ARTISTES AVEC LE PLUS DE STREAMS TOTAUX CUMULES}

On décide d'analyser ces musiques en fonction de leur clé musicale. Dans
le dataset, chaque clé est associée à un entier selon
\href{https://en.wikipedia.org/wiki/Pitch_class\#:~:text=\%5Bedit\%5D-,Pitch\%20class,-Pitch\%0Aclass}{cet
encodage}. Globalement, plus une clé est élevée, plus une chanson a
tendance à être aïgue. A l'inverse, une chanson à clé basse sera plus
grave.

Sur l'ensemble du dataset, la clé moyenne est :

\begin{verbatim}
## [1] 5.365503
\end{verbatim}

soit un \emph{Fa}.

\includegraphics{rapport_files/figure-latex/unnamed-chunk-3-1.pdf}

Si on s'intéresse uniquement aux 15 premières chansons, la clé moyenne
devient alors :

\begin{verbatim}
## [1] 3.866667
\end{verbatim}

plus proche d'un \emph{Mi} (soit une clé inférieure).

\includegraphics{rapport_files/figure-latex/unnamed-chunk-5-1.pdf}

En regardant les genres de toutes les musiques du dataset, on voit que
le genre avec la clé médiane la plus élevée est le rap. Les genres les
moins variés en terme de de clés musicales sont le rock et l'EDM
(Electronic Dance Music).

⚠️ Il est à noter que la classification du genre des musiques dans le
dataset peut parfois être hasardeuse.

\begin{quote}
Quelle est la durée moyenne des chansons par artiste ? Est-ce qu'il y a
des artistes qui ont des chansons plus longues ou plus courtes que la
moyenne ?
\end{quote}

L'idée à travers ces différentes interrogations est de pouvoir situer
les artistes en fonction de la durée moyenne de leurs chansons. Pour
cela, nous allons dans un premier temps voir la
\textbf{\emph{distribution}} de la durée moyenne de chansons à l'aide
d'un \textbf{\emph{histogramme}}, et ensuite faire la
\textbf{\emph{comparaison}} entre ces artistes. Avant cela, il est
important de noter que :

\begin{verbatim}
## [1] "la durée moyenne d'un song est de: 233.91s soit 3mn53s"
\end{verbatim}

\begin{verbatim}
## [1] "Le song le plus long dure: 517.12s soit 8mn37s"
\end{verbatim}

\begin{verbatim}
## [1] "Le plus court dure: 92.09s soit 1mn32s"
\end{verbatim}

\begin{center}\includegraphics{rapport_files/figure-latex/Histogramme-1} \end{center}

On constate à travers cet histogramme que toutes les chansons de notre
dataset durent plus de 100s soit plus d'une 1mn40s et qu'il y en a qui
vont jusqu'à plus de 400s soit plus de 6mn. Cependant malgré la faible
domination des songs dans la moyenne générale de durée, un certain
nombre d'artistes proposent une discographie de musiques plus longues.

Si l'on pousse la curiosité encore plus loin, on peut faire le top 10
des artistes qui produisent en moyenne plus de longues ou au contraire
plus de courtes musiques. Pour cela, nous, utiliserons un
\textbf{\emph{Bar\_Chart}} pour faire la \textbf{\emph{comparaison}}
entre ces artistes.

\begin{center}\includegraphics{rapport_files/figure-latex/Artist mean duration -1} \end{center}

Sur ces 2 graphiques, on peut voir que dans les deux tops 10, la moyenne
des artistes se détache largement de la durée moyenne du dataset et que
``\emph{Bob Marley \& The Wailers}'' fait plus de longues chansons
contrairement à ``\emph{Pouya}'' qui a la moyenne la plus basse.

⚠️ Il est important de relativisé, car il ne faut pas perdre de vue que
nous sommes en présence d'un dataset qui ne contient pas toute la
discographie de ces artistes.

Avec ces constats, on est en droit de se demander si la durée des songs,
n'a pas d'impact sur la popularité de la musique.

\begin{quote}
Les chansons plus longues ont-elles une popularité différente ?
\end{quote}

En tant que mélomane, on a tendance à croire que les chansons les plus
longues sont moins populaires en raison de la tendance actuelle des
artistes pour les musiques de courte durée. Afin d'affirmer ou
d'infirmer cette théorie, nous allons à l'aide d'un
\textbf{\emph{Scatter Plot}}, étudier la \textbf{\emph{relation}} entre
la durée et la popularité des différents songs du dataset.

\begin{verbatim}
## `geom_smooth()` using method = 'gam' and formula = 'y ~ s(x, bs = "cs")'
\end{verbatim}

\begin{center}\includegraphics{rapport_files/figure-latex/duration & popularity-1} \end{center}

À travers ce nuage de points, on observe :

\begin{itemize}
\tightlist
\item
  une forte concentration de songs ayant une durée comprise entre 200 et
  300 secondes et avec une popularité oscillant entre 60 et 80 \%.
\item
  les songs les plus, mais aussi les moins populaires se situent dans la
  même plage de durée,
\item
  les musiques les plus courtes et les plus longues partagent la même
  plage de popularité.
\item
  la dispersion du nuage de points par rapport à la droite de
  régression.
\end{itemize}

Tout ceci nous pousse à douter de l'existence d'une réelle relation
entre la popularité et la durée d'une musique. Dans le doute, on se
réfère au coefficient de corrélation qui est de :

\begin{verbatim}
## [1] -0.1568972
\end{verbatim}

Le résultat négatif traduit une corrélation négative, c'est-à-dire que
les deux variables évoluent en direction opposée, ce qui conforte nos
soupçons sur l'impopularité des longues chansons. Cette relation demeure
très faible et nous pensons que la taille de notre jeu de données ne
nous permet pas de s'assurer aisément de cette relation.

\begin{quote}
Quel genre de musique est le plus populaire sur Spotify/Youtube ?
\end{quote}

Dans le graphique ci-dessous, nous pouvons voir le classement des genres
de musiques par rapport au nombre moyen de stream (sur spotify et
youtube) que fait une musique de ce genre, ce qui donne donc un
classement de popularité. On choisi cette méthode de \emph{``nombre
moyen de stream que fait une musique''} pour catégoriser la popularité
des genres car nous ne pouvons pas simplement afficher un nombre total
de Stream pour chaque genre sachant qu'il n'y a pas un nombre égal de
musique par genre. Il faut diviser le nombre total de stream par genre
par le nombre de musique de ce genre, ce qui donne un nombre moyen de
stream pour chaque genre.

\includegraphics{rapport_files/figure-latex/unnamed-chunk-6-1.pdf}

La premiere chose que l'on remarque c'est que l'\textbf{EDM} (Electronic
Dance Music) est en tête de classement suivie par la \textbf{Pop} et les
musiques \textbf{Latin(o)}. Pourtant si on reprend les 15 musiques les
plus populaire et qu'on les classe par genre, on se rend compte que le
\textbf{Latin(o)} domine le classement, suivit par la \textbf{Pop} puis
enfin l'\textbf{EDM}. Comme le montre le graphique ci-dessous.

\includegraphics{rapport_files/figure-latex/unnamed-chunk-7-1.pdf}

Pour résumer ces deux graphiques, dans ce dataset, le genre de musique
le plus populaire est l'EMD (sur un nombre moyen d'écoute) mais parmi le
top 15 des musiques, le genre de musique le plus populaire est le
Latin(o). Alors, comment expliquer cette différence de résulat ?

Grace au diagramme Whisker Plot
\href{https://cdn2.boryl.fr/2020/11/3d5689a5-box-plot-boi\%CC\%82tes-a\%CC\%80-moustaches--300x223.png}{(définition)}
ci-dessous, on voit que le classement par la médiane nous donne encore
une information différente en mettant la Pop en première place. C'est
normal et même plus juste car la \textbf{moyenne est sensible aux
``outliers''}, tandis que la médiane est plus robuste à leur présence.
Si un petit nombre de valeurs extrêmes sont significativement plus
élevées que les autres valeurs de l'ensemble de données, elles peuvent
tirer la moyenne vers le haut, tandis que la médiane reste relativement
stable. De plus l'impact sur la moyenne peut varier en fonction de la
quantité de données et de la répartition des valeurs extrêmes.

\includegraphics{rapport_files/figure-latex/unnamed-chunk-8-1.pdf}

Comme on peut le voir dans le tableau ci dessous, le genre EDM contient
beaucoup moins de musique par rapport à la Pop ou au Latin(o), donc le
petit nombre de musiques extrêmes (en nombre de stream) à une grande
influence sur la moyenne de stream ce qui a tendance à fausser notre
classement initial.

\begin{verbatim}
##   Genre Nombre_de_musiques
## 1   edm                 89
## 2 latin                284
## 3   pop                385
## 4   r&b                310
## 5   rap                282
## 6  rock                598
\end{verbatim}

Que faut-il conclure pour cette question ? Le classement par la médiane
est plus pertinent, ce qui place donc la \textbf{Pop comme le genre le
plus populaire}. Pour le genre Latin(o), les musiques font généralement
moins de streams que l'EDM et la Pop sauf quelques cas particuliers qui
apparaissent dans notre top 15 des musiques les plus populaire.

⚠️ Il est à noter cette étude est basée sur les musiques présente dans
le dataset et pas sur toute les musiques existantes.

\begin{quote}
Est ce que l'energy d'une musique joue un rôle important dans sa
popularité ?
\end{quote}

C'est une question légitime, existe t'il une corrélation entre l'energie
d'une musique ainsi que sa popularité (total du nombre de streams sur
les 2 plateformes). Nous analysons cette question avec 2 graphiques
utilisant les mêmes données. Le 1er graphique est un scatter plot avec
une moyenne mobile du nombre de streams. Le 2nd est le même graphique
sans le nuage de points pour mieux analyser la moyenne mobile qui semble
constante sur le 1er.

\begin{Shaded}
\begin{Highlighting}[]
\FunctionTok{ggplot}\NormalTok{(data, }\FunctionTok{aes}\NormalTok{(}\AttributeTok{x =}\NormalTok{ energy, }\AttributeTok{y =}\NormalTok{ total\_streams)) }\SpecialCharTok{+} \FunctionTok{geom\_point}\NormalTok{() }\SpecialCharTok{+}
  \FunctionTok{geom\_smooth}\NormalTok{() }\SpecialCharTok{+}
  \FunctionTok{labs}\NormalTok{(}\AttributeTok{x =} \StringTok{"Energy"}\NormalTok{, }\AttributeTok{y =} \StringTok{"Total Streams"}\NormalTok{, }\AttributeTok{title =} \StringTok{"Correlation between Energy and Total Streams"}\NormalTok{) }\SpecialCharTok{+}
  \FunctionTok{theme\_minimal}\NormalTok{()}
\end{Highlighting}
\end{Shaded}

\begin{verbatim}
## `geom_smooth()` using method = 'gam' and formula = 'y ~ s(x, bs = "cs")'
\end{verbatim}

\includegraphics{rapport_files/figure-latex/unnamed-chunk-10-1.pdf}

On peut remarquer que la densité d'énergie avec le plus de musiques se
situe entre 0.4 et 1.

\begin{Shaded}
\begin{Highlighting}[]
\FunctionTok{ggplot}\NormalTok{(data, }\FunctionTok{aes}\NormalTok{(}\AttributeTok{x =}\NormalTok{ energy, }\AttributeTok{y =}\NormalTok{ total\_streams)) }\SpecialCharTok{+}
  \FunctionTok{geom\_smooth}\NormalTok{() }\SpecialCharTok{+}
  \FunctionTok{labs}\NormalTok{(}\AttributeTok{x =} \StringTok{"Energy"}\NormalTok{, }\AttributeTok{y =} \StringTok{"Total Streams"}\NormalTok{, }\AttributeTok{title =} \StringTok{"Correlation between Energy and Total Streams"}\NormalTok{) }\SpecialCharTok{+}
  \FunctionTok{theme\_minimal}\NormalTok{()}
\end{Highlighting}
\end{Shaded}

\begin{verbatim}
## `geom_smooth()` using method = 'gam' and formula = 'y ~ s(x, bs = "cs")'
\end{verbatim}

\includegraphics{rapport_files/figure-latex/unnamed-chunk-11-1.pdf}

L'évolution de la moyenne mobile est nettement plus analysable sans le
nuage de points. On peut remarquer que la moyenne du nombre de streams
augmente entre 0 et 0.75 d'energy pour ensuite diminuer entre 0.75 et 1.
Il y a une croissance de près de 25\% entre 0.25 et 0.75 d'energy et une
chute de 33\% du nombre de streams entre 0.75 et 1. On peut donc
conclure qu'il y a une certaine corrélation entre l'energy et le nombre
de streams.

\end{document}
